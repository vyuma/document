\documentclass[titlepage]{ltjsarticle}
% ページスタイル
\pagestyle{headings}
% 数式
\usepackage{amsmath,amsfonts}
\usepackage{bm}
% 数式
\usepackage{amsthm}
\usepackage{amssymb}
\usepackage{amsfonts}
\usepackage{bm}
\usepackage{derivative}
% 式番号を途中でふらない(参照するものだけ参照する)
\usepackage{mathtools}
\mathtoolsset{showonlyrefs=true}
% 数式をすべてdisplaystyleにする。
\everymath{\displaystyle}
\usepackage{bookmark}
\hypersetup{unicode,bookmarksnumbered=true,hidelinks,final}
% mysimpleboxの設定
\usepackage{tcolorbox}
\tcbuselibrary{skins}
\newtcolorbox{mysimplebox}[1]{%
colframe=black,colback=white,
coltitle=black,colbacktitle=white,
boxrule=0.8pt,arc=0mm,
fonttitle=\sffamily\bfseries,
enhanced,
attach boxed title to top left={xshift=10mm ,yshift=-3mm},
boxed title style={frame hidden},
title=#1}
% Pythonでのグラフ描画をインポートするために必要なパッケージ
\usepackage{graphicx}
\usepackage[utf8]{inputenc}
\usepackage{pgfplots}
\pgfplotsset{compat=1.18}
\begin{document}
\title{物性物理学}
\author{寺崎一郎}
\date{\today}
\maketitle  

\section{第3回}
前回の復習から、考える。
\begin{enumerate}
  \item 自由電子は\(e^{ikx}\)でエネルギーは、\(\frac{\hbar^2k^2}{2m}\)運動量は\(\hbar k\)となる。
  \item PBCCでは\(L=Na\)となり、
  \begin{equation}
    \varphi(x) = \varphi(x+L) = \varphi(x+Na)
  \end{equation}
  となり、\(k=\frac{2 \pi}{Na}\)の整数\(\frac{1}{\sqrt{L}}e^{ikx}\)となる、
  \item 第一BZについては、
  \begin{equation}
    -\frac{\tau}{a} \le k \le \frac{\pi}{a}
  \end{equation}
  となる。独立な\(k\)のセットである。
  \item \(a\)の並進対称性としては、
  \begin{align}
    \hat{T} \varphi(x) &= \varphi(x+a) = \varphi(x) \\
    \hat{T} \varphi(k) &= e^{ika} \varphi(k)
  \end{align}
  となる。同じ固有値\(e^{ika}\)をもつ\(k\)の組
  \begin{equation}
    -\frac{\pi}{a} \le k \le \frac{\pi}{a}
  \end{equation}
  ととると、
  \begin{align}
    k = k_0 + k_m \\
    k_m = \frac{2 \pi}{a}m
  \end{align}
  となる。
\end{enumerate}
\subsubsection{二次元の自由電子}
二次元の自由電子では、
\begin{equation}
  -\frac{\hbar^2}{2m} \left( \pdv[order=2]{}{x} + \pdv[order=2]{}{y} \right)\varphi(x.y) = \varepsilon \varphi(x,y)
\end{equation}
となり、\(\varphi(x,y) = X(x)Y(y)\)と仮定する。
すると、
\begin{equation}
  - \frac{\hbar^2}{2m} \pdv[order=2]{X}{x}Y(y) - \frac{\hbar^2}{2m} X(x) \pdv[order=2]{Y}{y} = \varepsilon X(x)Y(y)
\end{equation}
となることから、辺々を\(XY\)で割ると、
\begin{equation}
  X \propto e^{ik_x x}, Y \propto e^{ik_y y}
\end{equation}
となることから、\(\varepsilon_x = \frac{\hbar^2 k^2}{2m},\varepsilon_y = \frac{\hbar^2 k^2}{2m},\varepsilon_y\)となる。

ここから、
\begin{equation}
  \exp \left( \bm{k}\cdot \bm{\Gamma} \right) = \exp\left( k_x x + k_y y \right)
\end{equation}
となる。
エネルギーとしては、
\begin{equation}
  \varepsilon = \frac{\hbar^2}{2m}(k_x^2 + k_y^2)
\end{equation}
となる。
さらに、運動量は、
\begin{equation}
  \bm{p} = \hbar \bm{k} = (\hbar k_x, \hbar k_y)
\end{equation}
となる。
さらに、PBCは、
\begin{equation}
  \varphi(x,y) = \varphi(x+L,y) = \varphi(x,y+L)
\end{equation}
となることがわかる。\(e^{ik_xL}=e^{ik_yL}=1\)となる。ことから、規格化されて、
\begin{equation}
  k_x = \frac{2 \pi}{L}n_x, k_y = \frac{2 \pi}{L}n_y
\end{equation}
となり、規格化して、\(\int_{0}^{L}\int_{0}^{L}\odif{x,y}|\varphi|^2 =1\)となることから、
\(A=\frac{1}{L}\)をとり、
\begin{equation}
  \varphi(x,y) = \frac{1}{\sqrt{L^2}}e^{i\bm{k}\cdot \bm{r}}
\end{equation}
となるために、第一BZは、\(-\frac{\pi}{a}\le k_x \le \frac{\pi}{a},-\frac{\pi}{a}\le k_y \le \frac{\pi}{a}\)となる。

\subsubsection{\(a\)の並進対称性}

\(a\)の並進対称性は、
\begin{align}
  \hat{T}_x \varphi(x,y) &= \varphi(x+a,y) = \varphi(x,y) \\
  \varphi(x,y) & = \frac{1}{L}e^{i \bm{k}\cdot \bm{r}}\\
  \hat{T}_x \varphi(x,y) & = e^{ik_x a} \varphi(x,y)\\
  \varphi(x+a,y) & = \frac{1}{L} e^{i \bm{k}\cdot \bm{r}} \\
  & = e^{ik_x a} \varphi(x,y)
\end{align}
となることから、
第一BZ内の\(k_x\)に対して、
\begin{equation}
  k_x + k_{m_x}
\end{equation}
は、\(\hat{T}_x\)に対して、同じ固有値\(e^{ik_x a}\)を与える。
\begin{equation}
  K_{m_x} = \frac{2 \pi}{a}m_x
\end{equation}
ただし、\(m_x\)は整数である。どうように\(k_{m_y}=\frac{2\pi}{a}m_y\)が定義できる。

\subsubsection{逆格子}
実格子は\(a\)ずつの点であるが、逆格子は\(\frac{2\pi}{a}\)ずつの点のことである。

二次元の場合には、逆格子は\(-\frac{\pi}{a}\le k_x \le \frac{\pi}{a},-\frac{\pi}{a}\le k_y \le \frac{\pi}{a}\)が第一BZになり、次は、\(k_x = \pm \frac{2\pi}{a},k_y = \pm \frac{2 \pi}{a}\)wそれぞれ結んだひし形から第一BZを引いた値である。

さてもう少し抽象的なことをいおう。
まずは実格子に必要なベクトルのことを
\begin{equation}
  \begin{cases}
    \bm{a}_1 = a \bm{e}_x \\
    \bm{a}_2 = a \bm{e}_y
  \end{cases}
\end{equation}
となる。そして、逆格子は
\begin{equation}
  \begin{cases}
    \bm{b}_1 = \frac{2\pi}{a}\bm{e}_x = \frac{2\pi}{a} \bm{e}_{k_x} \\
    \bm{b}_2 = \frac{2\pi}{a}\bm{e}_y= \frac{2\pi}{a} \bm{e}_{k_y}
  \end{cases}
\end{equation}
となる。

\subsubsection{一般の二次元格子と逆格子}
一般の二次格子を拡張しておく。
\begin{equation}
  |\bm{a}_1 | = | \bm{a}_2 | 
\end{equation}
となり、\( \bm{a}_1\)と\(\bm{a}_2\)は直交しない。
対応する\(\bm{b}_1,\bm{b}_2\)はどう取ったらいいか?
一次元の場合には、
\begin{equation}
  \exp\left( ika \right) = \exp\left( i \left( k + k_m \right)a \right)
\end{equation}
となり、\(k_m = \frac{2\pi}{a}m\)となる、
二次元正方格子の場合には、
\begin{equation}
  \exp \left( i \bm{k} \cdot \bm{r} \right) \to \exp\left( \bm{k}\left( \bm{r} + \bm{a} \right) \right)  
\end{equation}
となる。ここから、\(\bm{K}\cdot \bm{a}\)が\(2 \pi\)の整数倍になるようにする。
ここで、\(\bm{a}_1\cdot \bm{b}_1 = 2\pi,\bm{a}_1 \cdot \bm{b}_2= \bm{a}_2 \cdot \bm{b}_1 = 0, \bm{a}_2 \cdot \bm{b}_2 = 2\pi\)となることを用いて、
\begin{align}
  \bm{k} & = n_1 \bm{b}_1 + n_2 \bm{b}_2 \\
  \bm{k}\cdot \bm{a} & = (m_1 \bm{b}_1 + m_2 \bm{b}_2) \cdot (n_1 \bm{a}_1 + n_2 \bm{a}_2) \\
   & = m_1 n_1 \bm{a}_1 \bm{b}_1 + m_2 n_2 \bm{a}_2 \bm{b}_2 + m_1 n_2 \bm{a}_1 \bm{b}_2 + m_2 n_1 \bm{a}_2 \bm{b}_1\\
   & = 2 \pi (m_1 n_1 + m_2 n_2)
\end{align}

これから、これを拡張すると、一般の\(\bm{a}_1,\bm{a}_2\)に対して、
\begin{equation}
  \begin{cases}
    \bm{a}_1 \cdot \bm{b}_1 = 2 \pi \\
    \bm{a}_1 \cdot \bm{b}_2 =\bm{a}_2 \cdot \bm{b}_1 =0 \\
    \bm{a}_2 \cdot \bm{b}_2 = 2 \pi
  \end{cases}
\end{equation}
となるように\(\bm{b}_1,\bm{b}_2\)となるように決める。

第三回の画像を見ながら
逆格子の世界には色がない。違う原子があったから撮って色が違うわけではない。色が互い違いにちがうときに、同じところに戻って来るのは\(2a\)である。

逆格子の世界に種類はないので、最小になるように選ぶ。


\end{document}