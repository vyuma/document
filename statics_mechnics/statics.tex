\documentclass[titlepage]{ltjsarticle}
% ページスタイル
\pagestyle{headings}
% 数式
\usepackage{amsmath,amsfonts}
\usepackage{bm}
% 数式
\usepackage{amsthm}
\usepackage{amssymb}
\usepackage{amsfonts}
\usepackage{bm}
\usepackage{derivative}
% 式番号を途中でふらない(参照するものだけ参照する)
\usepackage{mathtools}
\mathtoolsset{showonlyrefs=true}
% 数式をすべてdisplaystyleにする。
\everymath{\displaystyle}
\usepackage{bookmark}
\hypersetup{unicode,bookmarksnumbered=true,hidelinks,final}
% mysimpleboxの設定
\usepackage{tcolorbox}
\tcbuselibrary{skins}
\newtcolorbox{mysimplebox}[1]{%
colframe=black,colback=white,
coltitle=black,colbacktitle=white,
boxrule=0.8pt,arc=0mm,
fonttitle=\sffamily\bfseries,
enhanced,
attach boxed title to top left={xshift=10mm ,yshift=-3mm},
boxed title style={frame hidden},
title=#1}
% Pythonでのグラフ描画をインポートするために必要なパッケージ
\usepackage{graphicx}
\usepackage[utf8]{inputenc}
\usepackage{pgfplots}
\pgfplotsset{compat=1.18}
\begin{document}
\title{統計力学Ⅲ}
\author{大成誠一}
\date{\today}
\maketitle


簡単な小テストを毎回行う。時間があれば期末試験だけで行う。必修ではないので期末試験を受けなければ欠席扱いになる。
参考図書もシラバスにあるので

\section*{復習}
\subsection{グランドカノニカル分布}
開いた系の統計分布のことを開放系ともいい、外界とエネルギー、粒子のやりとりをおこなう系

対象とする開いた系Aと対象とする開いた系Bを考える。ここでAとBの相互作用は無視する。
この時にAとBを合わせた全系について考えるとそれが熱平衡になる。
この時に全系はミクロカノニカル分布になる。

系AがA粒子数がNでエネルギーがEだとする。
系Bが粒子数が\(N\)で、系が\(N_B\)だとする。エネルギーが\(E_B\)である。
全系は\(N+N_B=N_T\)となり、エネルギーは\(E+E_B=E_T\)となる。

ここで等確率の原理とはミクロカノニカル分布に従う全系の量子状態はすべて等しい確率で実現する。
エネルギーが一緒ならば同時に存在することが出来る。

Aが粒子数\(N\)で、エネルギー\(E\)の一つの量子状態にある確率を\(P(N,E)\)とする。
このとき、Bは粒子数\(N_T-N\)で、エネルギー\(E_T-E\)の量子状態にありその場合の数を
\(W_B(N_T-N,E_E-E)\)とすると、
等確率の原理から考えると
\begin{equation}
  P(N,E) \propto W_B(N_T-N,E_T-E)
\end{equation}
となる。

Bのエントロピーを\(S_B\)とすると、
\begin{equation}
  S_B = k\log W_B(N_B,E_B)
\end{equation}
となる。
依って、
\begin{equation}\label{eq:0.1}
  p(N,E) \propto \exp\left( \frac{1}{k_B}S_B(N_T-N,E_T-E) \right)
\end{equation}
となる。
外界BはAよりも十分大きいとして、
\begin{equation}
  N \ll N_T  ,\quad E \ll E_T
\end{equation}
となるために、
\begin{align}
  S_B (N_T-N,E_T-E) &= S_B(N_T,E_T) - \left( \frac{\partial S_B}{\partial N} \right)_{E_T}N - \left( \frac{\partial S_B}{\partial E} \right)_{N_T}E \\
  &  = S_B(N_T,E_T)  + \frac{\mu}{T} N - \frac{E}{T}
\end{align}
となる。
ここから、
\begin{equation}
  G = N@m  = U + PV - TS
\end{equation}
となり、
\begin{equation}
  S = \frac{U+PV-N \mu}{T}
\end{equation}
を用いた。よって、式\eqref{eq:0.1}より、\(p(N,E)\propto \exp\left( -\frac{1}{k_BT}(E-\mu N) \right)\)となる。規格化して、
\begin{equation}
  p(N,E) = \frac{1}{\Xi} \exp\left( -\frac{1}{k_BT}(E-\mu N) \right)
\end{equation}
として、大分配関数\(\Xi = \sum_N \sum_n \exp \left( -\frac{1}{k_BT} \left( E_n(N)-\mu N \right) \right)\)である。
この時に、\(E_n(N)\)は\(N\)粒子系の\(n\)番目のエネルギーである。

また分配関数
\begin{equation}
  Z_N = \sum_n \exp \left( -\frac{1}{k_BT}E_n(N) \right)
\end{equation}
となる。
この時に、
\begin{equation}
  \Xi = \sum_N \exp \left( \frac{\mu N}{k_BT} \right)Z_N = \sum_N \lambda^N Z_N
\end{equation}
となる。絶対活動度を\(\lambda = \exp \left( \frac{\mu}{k_BT} \right)\)とする。

一方粒子数は、
\begin{equation}
  \ev{N} = \frac{\sum_N N \lambda^N Z_N}{\sum_N \lambda^N Z_N} = \lambda \pdv{}{\lambda} \log \Xi
\end{equation}
となる。
またグランドポテンシャルは、
\begin{equation}
  \Omega = -k_BT \log \Xi
\end{equation}
となれば、\(\ev{N} = - \pdv{\Omega}{\mu}\)とも書ける。

ここで\(\Omega\)と熱力学との関係は?
そこで、\(\log \Xi\)を全微分してみよう!
この時の前提はAが開放系であるという設定のみ残っている。
\begin{align}
  \odif{(\log \Xi)} & = \sum_N \sum_n - \frac{N \mu-E_n(N)}{k_BT^2} \odif{T} \exp\left[ -\frac{E_n-N\mu}{k_BT} \right]\frac{1}{\Xi}  + \sum_N \sum_n \frac{\odif{(N\mu)}-\odif{(E)}_n(N)}{k_BT} \exp \left[ -\frac{E_n-N\mu}{k_BT} \right]\frac{1}{\Xi}\\
  & = -\frac{\ev{N}\mu}{k_BT^2} \odif{T} + \frac{\ev{E}}{k_BT^2}\odif{T} + \frac{\ev{\odif{(N\mu)}}}{k_BT} - \frac{\ev{\odif{E}}}{k_BT}\\
  & = \odif{\left( \frac{N \mu}{k_BT} \right)} + \frac{U}{k_BT^2} \odif{T} - \frac{\odif{'w}+\mu \odif{N}}{k_BT} \label{eq:0.2}
\end{align}
となる。ここで、\(\ev{E}=U\)と、\(\ev{\odif{E}}= \odif{'w}+ \mu \odif{N}\)を用いた。
ここで、\(\odif{E}_n(N)\)は\(N\)が一定の時に、エネルギー準位の変化なのでばね定数の変化等に対応する。
その平均値は加えられた仕事\(\odif{'W}\)に相当する。
一方\(N\)が変化するときには\(U\)の粒子数による変化\(\mu \odif{N}\)に対応する。

一方\(G=N\mu = U + PV - TS\)となるので、
\begin{align}
  \odif{\left( \frac{PV}{T} \right)} & = \odif{\left( \frac{G}{T} \right)} = 
  \odif{\left( \frac{N\mu}{T} \right)} - \odif{\left( \frac{U}{T} - S \right)} \\
  & = \odif{\left( \frac{N \mu}{T} \right)} + \left( \frac{U}{T^2} \right)\odif{T} - \frac{\odif{'w}+ \mu\odif{N}}{T} 
\end{align}
となる。\eqref{eq:0.2}と比較して、
\begin{equation}
  \Omega = -k_BT \ln \Xi = -PV = F-G 
\end{equation}
となる。ここで、\(\odif{'w}= -P \odif{V}\)となる。
一般化した力\(X_i\)による仕事は、
\begin{equation}
  \odif{'w} = \sum_i X_i \odif{x_i}
\end{equation}
となり、
\begin{equation}
  G = F - \sum_{i} X_i x_i 
\end{equation}
となる。

この時、一般化した仕事で考えると、
\begin{equation}
  \Omega = \sum_i X_ix_i 
\end{equation}
となる。

さらに、
\begin{equation}
  \Omega = F -G = F - \mu N 
\end{equation}
ここでは、\(N\)依存性から\(\mu\)依存性に変化したルジャンドル変換であるので、
\begin{equation}
  \odif{\Omega} = -S \odif{T} - P \odif{V} - N \odif{\mu}
\end{equation}
となる。

\subsubsection*{応用篇}
\(N\)粒子の理想気体は、
分配関数から、考えると分配関数\(Z_N\)となり、
\begin{equation}
  Z_N = \frac{1}{N!h^{3N}} \int \cdots \int \exp \left( - \frac{E}{k_BT} \right) \odif[order=3N]{k} \odif[order=3N]{q}
\end{equation}
となることがわかる。ここで、\(E = \frac{1}{2m}\sum^{N}_{k=1}(k^2_{ix}+k^2_{iy}+k^2_{iz})\)となる。
\begin{equation}\label{eq:0.3}\label{eq:0.3}
  Z_N = \frac{V^N}{N!h^{3N}} \left( 2 \pi m k_BT \right)^{\frac{3N}{2}}
\end{equation}
となる。


\paragraph{問題}
\(\lambda=e^{\frac{\mu}{k_BT}}\)となるとき、
\begin{align}
  \Xi &= \sum_N^{\infty} \lambda^{N} Z_N \\
      & = \sum_{N=0}^{\infty}\frac{1}{N!} \left[ \frac{V}{h^3}(2\pi m k_BT )^{\frac{3}{2}}\lambda \right]^N \\
      & = \exp \left[ \frac{V}{h^3}(2\pi m k_BT )^{\frac{3}{2}}\lambda \right]\\
      \frac{PV}{k_BT} & = \ln \Xi \\
      & = \frac{V}{h^3}(2 \pi m k_B T)^{\frac{3}{2}}\lambda
\end{align}
となる。よって、
\begin{equation}
  N = \lambda \pdv{}{\lambda} \ln \Xi = \frac{PV}{k_BT}  
\end{equation}
より状態方程式が得られる。


\subsection{粒子数の揺らぎと等温圧縮率}
大分配関数\(\Xi\)となり、
\begin{equation}
  \Xi = \sum_{N=0}^{\infty}\lambda^N Z_N 
\end{equation}
となり、\(\lambda=e^{\frac{\mu}{k_BT}}\)に対して、
\begin{align}
  \left( \pdv{\ln \Xi}{\mu} \right) & = \frac{1}{\Xi} \sum_N \frac{N}{k_BT}\lambda^N Z_N\\
  & = \frac{1}{k_BT} \ev{N} 
\end{align}
となる。
さてさらにもう一度微分してあげると、
\begin{align}
  \left( \pdv[order=2]{\ln \Xi}{\mu} \right)_{T,N} &= \frac{1}{\Xi}\left( \frac{N}{k_BT} \right)^2 \sum_N \lambda^N Z_N - \left( \frac{1}{\Xi} \sum_N \frac{N}{k_BT}\lambda^N Z_N \right)^2\\
  & = \frac{1}{(k_BT)^2} \left( \ev{N^2} - \ev{N}^2 \right)
\end{align}
のような分散のような関係を得ることが出来る。
依って、
\begin{align}\label{eq:0.4}
  \left(\pdv{\ev{N}}{\mu}\right)_{T,V} & = \beta (\ev{N^2} - \ev{N}^2)\\
  & = \beta (\varDelta N)^2
\end{align}
となる。ここで、\(\beta = \frac{1}{k_BT}\)とし,\(\varDelta N = \sqrt{\ev{N^2} - \ev{N}^2}\)とした。
さて、統計力学のアプローチを考えた。

さてここからは熱力学からどう考えるか?について考察する。
熱力学から、
\begin{equation}
  G = N \mu(P,T)
\end{equation}
であることがわかる。これを微分すると、
\begin{equation}
  \odif{G} = N \odif{\mu} + \mu \odif{N}
\end{equation}
と書くことができる。また、
\begin{equation}
  \odif{G} = -S \odif{T} - V \odif{P} + \mu \odif{N}
\end{equation}
となる。

また、\(\odif{G}= -S \odif{T} +V \odif{P} + \mu \odif{N}\)と合わせて、Gibbs-Duhemの関係式が得られる。
\begin{equation}
  N \odif{\mu} + S \odif{T} - V \odif{P} = 0
\end{equation}

さらにこの式から、
\begin{equation}
  N \left( \pdv{\mu}{V} \right)_{T,N} =V\left( \pdv{P}{V} \right)_{T,N}
\end{equation}
と
\begin{equation}
  \left( \pdv{P}{\mu} \right)_{T,V} = \frac{N}{V}
\end{equation}
という関係式が得られる。

またヘルムホルツの自由エネルギーの関係の式から、
\begin{equation}
  \pdv{F}{N,V} = \pdv{F}{V,N} 
\end{equation}
から、\(\odif{F}= -S \odif{T} -P \odif{V} + \mu \odif{N}\)に対して、
\begin{equation}
  \left( \pdv{\mu}{V} \right)_{T,N} = - \left( \pdv{P}{N} \right)_{T,V} = - \left( \pdv{P}{\mu} \right)_{T,V}\left( \pdv{\mu}{V} \right)_{T,V}
\end{equation}
となることがわかる。

上の二式と合わせることにより、
\begin{equation}
  \left(\pdv{\mu}{V}\right)_{T,V} = - \frac{V^2}{N^2}\left( \pdv{P}{V} \right)_{T,N}
\end{equation}
となる。

これをまとめると、\(  \left( \pdv{\mu}{V} \right)_{T,N} = - \left( \pdv{P}{N} \right)_{T,V} = - \left( \pdv{P}{\mu} \right)_{T,V}\left( \pdv{\mu}{V} \right)_{T,V}\)を用いて、
\begin{equation}
  \left(\pdv{N}{\mu}\right)_{T,V} = - \frac{N^2}{V^2}\left( \pdv{V}{P} \right)_{T,N}
\end{equation}
となる。

依って、式\eqref{eq:0.4}より、
\begin{equation}
  -\frac{N^2}{V^2} \left( \pdv{V}{P} \right)_{T,V} = \beta (\varDelta N)^2
\end{equation}
となる。ここで等しい等温圧縮率\(\kappa_T = - \frac{1}{V}\left( \pdv{V}{P} \right)_{T,N}\)を用いることにより、
\begin{equation}
  \pdv{(\varDelta N)^2}{N} = \frac{Nk_BT}{V}\kappa_T 
\end{equation}
となることがわかる。

理想気体の時考えると、
\begin{equation}
  \kappa_T = -\frac{1}{V} \left(  -\frac{Nk_BT}{p^2} \right) = \frac{1}{P}
\end{equation}
となる。よって、
\begin{equation}
  \frac{\ev{\varDelta N}^2}{N} = 1  \Rightarrow \frac{\varDelta N}{\ev{N}} = \frac{1}{\ev{N}^{\frac{1}{2}}}
\end{equation}
となることがわかる。
つまり。\(\ev{N}\)が非常に大きいとき\(\varDelta N\)は無視できるようになる。


\section{量子統計力学(グランドカノニカル分布)}
\subsection{量子論}
一粒子状態\(\varphi_j\)のエネルギー準位を\(\varepsilon_j\)とする。
その量子様態にある粒子数を\(n_j\)とする。

\paragraph{粒子間の相互作用がない場合}

\begin{equation}\label{eq:1.1}
  \begin{cases}
    \text{全エネルギー} : & E = \sum_j n_j \varepsilon_j \\
    \text{全粒子数} : & N = \sum_j n_j
  \end{cases}
\end{equation}
となる。

\paragraph{量子の統計性}
量子はボーズ粒子やフェルミ粒子のような入れ替えの作用により変化する統計性が二つ存在することがわかる。
\(N\)粒子系の波動関数を\(\psi(r_1,r_2,\ldots ,r_N)\)となることがわかる。

同種の二粒子を入れ替えた状態は量子力学的に同等であるために、\(C\)を定数として、
\begin{equation}
  \psi(\dots,r_l,\dots,r_m,\dots) =C \psi(\dots,r_m,\dots,r_l,\dots)
\end{equation}
再び系が入れ替えるともとに戻るはずなので\(C^2=1\)より、
\begin{equation}
  C=  \pm 1 
\end{equation}


\paragraph{二粒子系}
二流紙の波動関数では、
\begin{equation}
  \psi^{\pm}_{12} = \frac{1}{\sqrt{2}}\left( \psi_1(r_1)\psi_2(r_2) \pm \psi_1(r_2)\psi_2(r_1) \right)
\end{equation}
となる。

フェルミ粒子の時に\(\alpha=\beta\)とすると、\(\psi^{-}_{\alpha,\alpha}(r_1,r_2)=0\)となる。
このために、電子であるとパウリの排他率が成り立つ。

つまり同一の量子状態を二粒子で占有することが出来ない。これをパウリの排他率という。

意向からは量子状態に対する統計力学を考える。
分配関数から、
\begin{align}
  Z_N & = \sum_i e^{\beta E_i} \\
  & = \sum_{\{n_j\}} e^{-\beta \sum_j n_j \varepsilon_j} \\
\end{align}
となる。

大分配関数では、
\begin{align}
  \Xi & = \sum_{N=0}^{\infty}e^{\beta \mu N} Z_N \\
  & = \sum_{N=0}^{\infty}\sum_{\{n_j\}}^{}e^{\beta \mu \sum_j n_j} e^{-\beta \sum_j n_j \varepsilon_j} \\
  & = \sum_{n_1}\sum_{n_1}\dots \prod_j e^{\beta \mu n_j} e^{-\beta n_j \varepsilon_j} \\
  & = \sum_{n_1} e^{\beta(\mu - \varepsilon_1)n_1}\sum_{n_2} e^{\beta(\mu - \varepsilon_2)n_2}\dots \\
  & = \prod_j \sum_{n_j} e^{\beta(\mu - \varepsilon_j)n_j} 
\end{align}
となる。

全系の量子状態はどの量子状態にいくつ粒子がいるかで決まる。
同種な粒子は区別できないという原理を置く。

この時、ボーズアインシュタイン統計について考えると、
\begin{description}
  \item[ボーズアインシュタイン統計] 一つの順位に任意個の粒子が入るとする。光子やフォノン
  \begin{equation}
    \Xi = \prod_j \sum_{n_j}^{\infty}e^{\beta(\mu-\varepsilon_j)n_j} = \prod_j \frac{1}{1-e^{\beta(\mu-\varepsilon_j)}}
  \end{equation}
  \item[フェルミディラック統計] フェルミディラック統計は、\(n_j=0,1\)となる。
  人等の順位に対して一項の粒子まで入れるこれをフェルミ粒子、電子陽子、中性子などがあり、
  \begin{equation}
    \Xi = \prod_j \sum^1_{n_j=0} e^{\beta(\mu - \varepsilon_j)n_j} = \prod_j \left[ 1 + e^{\beta(\mu-\varepsilon_j)} \right]
  \end{equation}
\end{description}
維持用のことをまとめると、
\begin{equation}
  \Xi = \prod_j \left[ 1 \mp e^{\beta(\mu-\varepsilon_j)} \right]^{\mp 1}
\end{equation}
となりグランドポテンシャルが
\begin{equation}
  \Omega = -k_BT \sum_j \ln \left[ 1 \mp e^{\beta(\mu-\varepsilon_j)} \right]^{\mp} = \sum_j \Omega_j 
\end{equation}
となり、
\begin{equation}
  \ev{N} = - \pdv{\Omega}{\mu} = \sum_j \frac{1}{e^{\beta(\varepsilon_j-\mu)} \mp 1}
\end{equation}
となる。


エネルギー状態が\(\varepsilon\)にある平均粒子数を考える。

\begin{equation}
  \begin{cases}
    n(\varepsilon)  =  \frac{1}{e^{(\varepsilon-\mu)/k_BT} \mp 1} & \text{ボーズ粒子} \\
    n(\varepsilon)  =  \frac{1}{e^{(\varepsilon-\mu)/k_BT} + 1} & \text{フェルミ粒子}
  \end{cases}
\end{equation}
となる。

問題)ボーズ粒子とフェル分布を導出せよ。

高温極限としては、高温の時\(n(\varepsilon),f(\varepsilon)\ll 1\)で、\(\mu<0\) かつ\(-\beta \mu \gg 1\)となっており\(e^{-\beta \mu} \gg 1\)この時、
\begin{equation}
  n(\varepsilon) = f(\varepsilon) \sim e^{-\beta(\varepsilon-\mu)}
\end{equation}
つまり古典論と一致に量子性が現れない。

\paragraph{古典理想気体と量子の違い}
古典と量子の違いは何か?という問いに対して1粒子分配関数を考えると、
\begin{equation}
  Z = \frac{V}{h^3} (2 \pi mk_BT)^{\frac{3}{2}}
\end{equation}
となる。

古典形での\(Z_N = \frac{1}{N!}Z_1^N\)は量子では間違いである。
古典論での過程は、
\begin{enumerate}
  \item 各粒子は同じ状態にならない これはボーズ粒子と矛盾する
  \item 各粒子の状態は独立に決まる これはパウリの排他率と矛盾する
\end{enumerate}

\section{理想フェルミ粒子}
\subsection{フェルミ分布}
エネルギーがその一粒子状態の平均粒子数を考えると、
\begin{equation}
  f(\varepsilon) = \frac{1}{e^{(\varepsilon-\mu)/k_BT}+1}
\end{equation}
となる。これをフェルミ分布という。


\(T=0\)での\(\mu=\mu_0\equiv \varepsilon_F\)となり\(\varepsilon_F\)のことをフェルミエネルギーと呼ぶ。
グラフで書くと、\(\varepsilon_F\)でのステップ関数のようになるが、\(T>0\)でのフェルミ分布は滑らかになる。
温度による\(f(\varepsilon)\)の広がり幅としては\(T \)となる。
テストで毎回フェルミ分布関数を出している。
これを書けるようにしよう。

\subsection{自由電子模型}
1価金価\(Na\)がぞんざいして、\((1s)^2,(2s)^2,,(2p)^6,(3s)^1\)となる。
この時に、\(3s\)電子としては伝導電子として結晶中を自由に動き回る。
自由電子模型としては、
\begin{equation}
  -\frac{\hbar^2}{2m}\left( \pdv[order=2]{}{x} + \pdv[order=2]{}{y} + \pdv[order=2]{}{z} \right) \psi = E \psi 
\end{equation}
となるので、周期的境界条件を生かしてあげる必要がある。

一片の長さを\(L\)として、周期的境界条件を満たすとする。

\begin{equation}
  \psi = e^{i \frac{2\pi}{L}(n_x x + n_y y + n_z z)}
\end{equation}
となり、\(n_x,n_y,n_z\)となり、波動関数の形が変わらない。

\begin{equation}
  E = \frac{\hbar^2}{2m} \left( \frac{2\pi}{L} \right)(n_x^2+n_y^2+n_z^2)
\end{equation}
となり、エネルギー\(E\)より小さい状態の数をカウントする。

\begin{equation}
  n_x^2 + n_y^2 + n_z^2 \leq \frac{2m}{\hbar^2}\left( \frac{L}{2\pi} \right)^2 E 
\end{equation}
となることから、割り算してしまえばこれ以上小さいという数を数えれば、状態の数に対応する。その整数値の数を数えればよい。
その整数値は体積を計算してあげればいい。

そのために、半径が\(\left[ \frac{2m}{\hbar^2}\left( \frac{L}{2\pi} \right)^2 E \right]\)の球の内部における格子点の数について考える。ここから、体積\(V\)は、
\begin{equation}
  V = \frac{4\pi}{3}\left[ \frac{2m}{\hbar^2}\left( \frac{L}{2\pi} \right)^2 E \right]^{\frac{3}{2}}
\end{equation}
となる。
エネルギーが\(E\)と\(E+\odif{E}\)の間にある状態の数\(g\)はスピン縮重度として
\begin{equation}
  D(E) \odif{E} = \frac{2\pi V }{\hbar^3} \sqrt{2mE}g \odif{E}
\end{equation}
となる。\(g\)については\(g=(2s+1)\)で電子のばあいは\(s=\frac{1}{2}\)となる。
全電子数\(N\)は、
\begin{equation}
  N = \int^{\infty}_{0} D(E)f(E) \odif{E}
\end{equation}
となる。
この時にはエネルギーが負にならないので、
下限をゼロとして、\(T=0\)の場合には、
\begin{align}
  N & = \int^{\infty}_{0} f(E)D(E) \odif{E} \\
  & = \int_{0}^{\varepsilon_F} \frac{2\pi g V}{h^3} (2m)^{\frac{3}{2}} E^{\frac{1}{2}} \odif{E} \\
  & = \frac{4\pi g V }{3h^3} (2m)^{\frac{3}{2}} \varepsilon_F^{\frac{3}{2}}
\end{align}
となるために、
\begin{equation}
  \varphi_F = \frac{1}{2m}\left( \frac{3h^3N}{4\pi gV} \right)^{\frac{2}{3}} = \frac{h^2}{2m}\left( \frac{3}{4\pi g}n \right)^{\frac{2}{3}}
\end{equation}
となることから、全エネルギーの場合には、
\begin{align}
  U & = \int_{0}^{\infty} E D(E)f(E) \odif{E} \\
  & = \int_{0}^{\varepsilon_F} \frac{2\pi g V}{h^3 }(2m)^{\frac{3}{2}} E^{\frac{3}{2}} \odif{E} \\
  & = \frac{4\pi gV}{5h^3} (2m)^{\frac{3}{2}}\varepsilon_F^{\frac{5}{2}}\\
  & = \frac{3}{5}\varepsilon_F N
\end{align}
依って一粒子あたりの運動エネルギーは\(\frac{3}{5}\varphi_F\)となる。

また演習問題より、\(PV = \frac{2}{3}U\)となるために、
\begin{equation}\label{eq:2.2}
  P = \frac{3}{5}\varepsilon_F n = \frac{h^2}{5m} \left( \frac{3}{4\pi g} \right)^{
    \frac{2}{3}}n^{\frac{5}{3}}
\end{equation}
となり、これは\(T=0\)の時の式になる。


\paragraph{フェルミ温度}
熱的ドブロイ波長では、
\begin{equation}
  \lambda_T = \frac{h}{(2\pi m k_BT)^{\frac{1}{2}}}
\end{equation}
となる\(\lambda_T\)は\(k_BT\)ほどのエネルギーを持つ粒子の波動関数の広がりを表す。
\begin{equation}
  \begin{cases}
    \lambda_T \gg \left( \frac{V}{N} \right)^{\frac{1}{3}} & \text{量子力学的効果が無視できる} \\
    \lambda_T \ll \left( \frac{V}{N} \right)^{\frac{1}{3}} & \text{量子力学的効果が無視できない}
  \end{cases}
\end{equation}
さて、\(\lambda_T > \left( \frac{V}{N} \right)^{\frac{1}{3}}\)は、
\(k_BT < \frac{h^2}{2\pi m}n^{\frac{2}{3}}\sim \varepsilon_F\)よって、\(T<T_F\equiv \frac{\varphi_F }{k_B}\)となる。この時、この時量子性が重要になる。

\paragraph{電子間相互作用}
電子一項あたりの電子間相互作用は\(e^2\left( \frac{N}{V} \right)^{\frac{1}{3}}\propto n^{\frac{1}{3}}\)となる。
一方一粒子あたりの運動エネルギーは\(\frac{3}{5}\varepsilon_F \propto n^{\frac{2}{3}}\)である。\(n\)は電子密度である。
以上のような関係性から、電子間相互作用を運動エネルギーで割ってあげよう。
この時にはどのようになるだろうか?
\begin{equation}
  \text{電子間相互作用} / \text{運動エネルギー} \propto n^{-\frac{1}{3}}
\end{equation}
電子の密度が高濃度になればなるほど電子間相互作用が運動エネルギーよりも小さくなることがわかる。
依って高密度ほど電子管相互作用が相対的に小さくなる。

\paragraph{等温圧縮率\(\kappa_T\)}について考えよう。
まず\eqref{eq:2.2}より、
\begin{equation}
  P = \frac{2}{5}\varepsilon_F n \propto V^{-\frac{5}{3}}
\end{equation}
であり、
\begin{equation}
  \frac{1}{\kappa_T} - V \left( \pdv{P}{V} \right)_{T,N} =  \frac{5}{3}P = \frac{2}{3}\varepsilon_F n 
\end{equation}
となる。



\paragraph{ナトリウムの原子で計算}
原子体積が\(24 \mathrm{cm^3/mol}\)であり、\(N=6.02\times 10^{23} \mathrm{/mol}\)
\(m= 9.107\times 10^{-31} \mathrm{kg}\)\(h=6.62\times 10^{-34}\mathrm{J\cdot s}\)
\(k_B = 1.38 \times 10^{-23} \mathrm{J/K}\)より、
\begin{equation}
  \varepsilon_F = 5.01 \times 10^{-19} \mathrm{J} = 3.13 \mathrm{eV}
\end{equation}
となり\(T_F = 3.63 \times 10^4 K\)であり、室温\(\sim 300 \mathrm{K}\)になる。

\subsection{状態密度\(D(E)\)}
ここで、
\begin{equation}
  \sum_k \to  \left( \frac{L}{2\pi} \right)^3 \int_{0}^{\infty} 4 \pi k^2 \odif{k}
\end{equation}
となる。\(\bm{k} = \frac{2\pi}{L}\left( n_x,n_y,n_z \right)\)になるとき、
\begin{equation}
  \sum_k \to \left( \frac{L}{2\pi} \right)^2 \int^{\infty}_{0}2\pi k \odif{k}
\end{equation}
となる。
一次元の時には、
\begin{equation}
  \sum_k \to \left( \frac{L}{2\pi} \right) \int^{\infty}_{0} \odif{k}
\end{equation}

また\(D(E)\)を用いると、
\begin{equation}
  \sum_k \to \int_{0}^{\infty}D(E) \odif{E}
\end{equation}

\paragraph{自由粒子の場合}
自由粒子の場合には、
\begin{equation}
  E = \frac{\hbar^2 k^2}{2m} \quad  \odif{E} = \frac{\hbar^2 k}{m} \odif{k}
\end{equation}
となり、3次元の場合には、
\begin{align}
  \left( \frac{L}{2\pi} \right)^3 4\pi^2k^2 \odif{k} & = \left( \frac{L}{2\pi} \right)^3 4 \pi \frac{2mE}{\hbar^2} \frac{m}{\hbar^2} \left( \frac{\hbar^2}{2mE} \right)^{\frac{1}{2}} \odif{E} \\
  & = \left( \frac{L}{2\pi} \right)^3 4 \pi \frac{2mE}{\hbar^2} \frac{m}{\hbar^2} \left( \frac{2m E}{\hbar^2} \right)^{\frac{1}{2}} \odif{E} \\
\end{align}
として、
\begin{equation}
  D(E) = \left( \frac{L}{2\pi} \right)^3 \frac{4\pi m}{\hbar^3} (2mE)^{\frac{1}{2}} \times g 
\end{equation}
となる。

二次元の場合には、
\begin{equation}
  D(E) = \left( \frac{1}{2\pi} \right)^2 \frac{2\pi m}{\hbar^2} \times g 
\end{equation}

一次元の場合には、
\begin{equation}
  \frac{L}{\pi} \odif{k} = \frac{L}{\pi} \frac{m}{\hbar^2} \left( \frac{\hbar^2}{2mE} \right)^{\frac{1}{2}} \odif{E} = \frac{L}{\pi} \left( \frac{m}{2 \hbar^2} \right)^{\frac{1}{2}}E^{-\frac{1}{2}} \odif{E}
\end{equation}
となり、
\begin{equation}
  D(E) = \frac{L}{\pi} \left( \frac{m}{2\hbar^2} \right)^{\frac{1}{2}} E^{-\frac{1}{2}} \times g
\end{equation}

\paragraph{問題}
2-3節の\(D(E)\)を導出せよ。


\subsection{フェルミ気体の有限温度の性質}
フェルミ分布関数については、
\begin{equation}
  f(\varepsilon) = \frac{1}{e^{(\varepsilon-\mu)/k_BT}+1}
\end{equation}
一般の物理量\(A(\varepsilon)\)の期待値を計算すると、
\begin{equation}
  \ev{A}_T = \int_{-\infty}^{\infty} A(\varepsilon)f(\varepsilon)D(\varepsilon) \odif{\varepsilon}
\end{equation}
となることがわかる。ここで、\(f(\varepsilon)\)は一状態の電子数である。
さてこの時の温度依存性は
\begin{align}
  \ev{A}_T & = \int_{-\infty}^{\infty} A(\varepsilon)f(\varepsilon)D(\varepsilon) \odif{\varepsilon}\\
  & = \ev{{A}}_{T = 0} + a T^2 + O(T^4)
\end{align}
となる。
\(f_B(T)\)が幅\(k_BT\)に広がる。\(\mu\)の温度変化については、図のようになる。


\tikzset{every picture/.style={line width=0.75pt}} %set default line width to 0.75pt        

\begin{tikzpicture}[x=0.75pt,y=0.75pt,yscale=-1,xscale=1]
%uncomment if require: \path (0,300); %set diagram left start at 0, and has height of 300

%Shape: Axis 2D [id:dp07737089035535272] 
\draw  (229,243.55) -- (489,243.55)(255,82) -- (255,261.5) (482,238.55) -- (489,243.55) -- (482,248.55) (250,89) -- (255,82) -- (260,89)  ;
%Curve Lines [id:da7783880358702222] 
\draw    (307,149) .. controls (362,144.5) and (334,237.5) .. (394,233.5) ;
%Straight Lines [id:da19423447951803552] 
\draw    (221,149) -- (307,149) ;
%Straight Lines [id:da4400596775089036] 
\draw    (394,233.5) -- (480,233.5) ;

% Text Node
\draw (367,248.4) node [anchor=north west][inner sep=0.75pt]    {$\mu $};
% Text Node
\draw (497,238.4) node [anchor=north west][inner sep=0.75pt]    {$T$};
% Text Node
\draw (215,61.4) node [anchor=north west][inner sep=0.75pt]    {$f( \varepsilon )$};


\end{tikzpicture}


ゾンマーフェルトの展開公式としては、
任意関数\(g(\varepsilon)\)として、
\begin{equation}
  G(\varepsilon) = \int_{-\infty}^{\varepsilon}g(\varepsilon') \odif[order=3]{\varepsilon'}
\end{equation}
となり、\(f(\infty)=0,G(-\infty)=0\)となる。
\begin{align}
  \int_{-\infty}^{\infty}\odif[order=3]{\varepsilon} g(\varepsilon) f(\varepsilon)  & = \left[ G(\varepsilon)f(\varepsilon) \right]^{\infty}_{-\infty} - \int_{-\infty}^{\infty}G(\varepsilon) \pdv{f(\varepsilon)}{\varepsilon}\odif{\varepsilon} \\
  & = G(\infty)f(\infty) - G(-\infty)f(-\infty) + \int_{-\infty}^{\infty}G(\varepsilon)\left( -\pdv{f}{\varepsilon} \right) \odif{\varepsilon}\\
  & = \int_{-\infty}^{\infty}G(\varepsilon)\left( -\pdv{f}{\varepsilon} \right) \odif{\varepsilon}
\end{align}
となる。
ここから、
\begin{equation}
  -\pdv{f}{\varepsilon} = \frac{\beta e^{\beta(\varepsilon-\mu)}}{\left[ e^{\beta(\varepsilon-\mu)} +1\right]^2} = \frac{\beta}{4}\frac{1}{\cosh\left( \frac{\beta}{2}(\varepsilon-\mu) \right)}
\end{equation}
となる。

\(-\left( \pdv{f}{\varepsilon} \right)\)の性質は以下のようになる。
\begin{equation}\label{eq:2.3}
  -\pdv{f}{\varepsilon} = 
  \begin{cases}
    \int_{-\infty}^{\infty}\left( -\pdv{f }{\varepsilon} \right)\odif{\varepsilon} = 1 \\
    \left( -\pdv{f}{\varepsilon} \right)_{T=0} = \delta(\varepsilon-\mu)\\
    \int_{-\infty}^{\infty}\left( - \pdv{f}{\varepsilon} \right)(\varepsilon-\mu) \odif{\varepsilon} = 0\\
    \int_{-\infty}^{\infty}\left( -\pdv{f}{\varepsilon} \right)(\varepsilon-\mu)^2 \odif{\varepsilon} = \frac{\pi^2}{3}(k_BT)^2
  \end{cases}
\end{equation}
となる。

\paragraph{問題}
式\eqref{eq:2.3}の導出をおこなおう!!


\(G(\varepsilon)\)を\(\varepsilon=\mu\)の周りで二次まで展開することをする。
テイラー展開をおこなおう!テイラー展開の二時まで行うと、
\begin{equation}
  G(\varepsilon) = G(\mu) + (\varepsilon-\mu) G'(\varepsilon) + \frac{1}{2}(\varepsilon-\mu)^2G''(\varepsilon) + \cdots 
\end{equation}
となり、
\begin{equation}\label{eq:2.4:ゾンマーフェルト展開公式}
  \int_{-\infty}^{\infty} \odif{\varepsilon} G(\mu) + \frac{\pi^2}{6}(k_BT)^2 G''(\mu) + \cdots 
\end{equation}
となる。これをゾ―マンフェルト展開公式という。

\(  \ev{A}_T = \int_{-\inftyは式}^{\infty} D(\varepsilon)A(\varepsilon)f(\varepsilon) \odif[order=3]{\varepsilon}\)となることから、\eqref{eq:2.4:ゾンマーフェルト展開公式}で\(g(\varepsilon)=D(\varepsilon)A(\varepsilon)\)としたものであり、
\begin{equation}\label{eq:2.5}
  \ev{A}_T = \ev{A}_{T=0} + \frac{\pi^2}{6}(k_BT)^2 G''(\mu)+ \cdots
\end{equation}
ここで、\(\mu=\mu_0+\Delta \mu(T)\)として低温のために\(\varDelta \mu (T)\)が小さいとする。
\begin{equation}
  G(\mu) = \int_{-\infty}^{\mu_0} \odif{\varepsilon} g(\varepsilon) + \int_{\mu_0}^{\mu_0+\delta \mu}\odif{\varepsilon}g(\varepsilon) = \ev{A}_{T=0} + D(\mu_0)A(\mu_0)\Delta \mu(T) + \cdots
\end{equation}
となる。


\eqref{eq:2.5}を\(T\)2次までで展開すると、
\begin{equation}\label{eq:2.6}
  \ev{A}_T = \ev{A}_{T=0} + \frac{\pi^2}{6}(k_BT)^2 D(\mu_0)A(\mu_0)\varDelta \mu(T) + \frac{\pi^2}{6}(k_BT)^2 \pdv{}{\varepsilon}(D(\varepsilon)A(\varepsilon))\Big|_{\varepsilon=\mu_0} + O(T^2)
\end{equation}
となる。
この時粒子数\(N(T)\)を計算する。

\paragraph{粒子数を計算する}
粒子数\(N(T)\)を計算しよう。\(A(\varepsilon)=1\)である。
式\eqref{eq:2.6}より\(T\)の二次まで求めると、
\begin{equation}
  N(T) \sim N(0) + D(\mu_0)\Delta \mu(T) + \frac{\pi^2}{6}(k_BT)^2 D'(\mu_0)
\end{equation}
となり、粒子数は温度に依らないために、\(N(T)=N(0)\)依って、
\begin{equation}\label{eq:2.7}
  \varDelta \mu(T) = -\frac{D'(\mu_0)}{D(\mu_0)}\frac{\pi^2}{6}(k_BT)^2
\end{equation}
となる。


3次元自由粒子の場合には、\(D(\varepsilon)=C_0\varepsilon^{\frac{1}{2}}\)
\begin{equation}
  \varDelta \mu(T) = -\frac{1}{2\mu_0} \frac{\pi^2}{6}(k_BT)^2 = -\frac{\pi^2}{12\mu_0}(k_BT)^2
\end{equation}
同様に1次元の場合には、
\begin{equation}
  D(\varepsilon) = C_0\varepsilon^{-\frac{1}{2}} , \quad \varDelta \mu(T) = \frac{\pi^2}{12 \mu_0}(k_BT)^2
\end{equation}
となる。二次元の場合は異なる方法で計算する必要があり、
\begin{equation}
  \varDelta \mu(T) = -k_BT e^{-\frac{\mu_0}{k_BT}}
\end{equation}
となる。

\paragraph{エネルギー\(U(T)\)を計算する}
エネルギー\(U(T)\)は、
\begin{equation}
  U(T) \sim U(T=0) + D(\mu)\mu_0 \varDelta \mu(T) + \frac{D(\varepsilon)\varepsilon}{\varepsilon}\Big|_{\mu_0}\frac{\pi^2}{6}(k_BT)^2
\end{equation}
\eqref{eq:2.7}より、
\begin{equation}
  D(\mu_0)\mu_0 \varDelta \mu(T) = -\mu_0 D'(\mu_0)\frac{\pi^2}{6}(k_BT)^2
\end{equation}
と用いて、
\begin{equation}
  U(T) \sim U(T=0) + D(\mu_0) \frac{\pi^2}{6}(k_BT)^2 
\end{equation}
依ってフェルミ気体の低温比熱は
\begin{equation}
  C = \pdv{U}{T} = \frac{\pi^2}{3}k_BT^2 D(\mu_0) T = \gamma T 
\end{equation}
となり比熱係数は\(\gamma = \frac{\pi^2}{3}k_B^2 D(\mu)\)

\paragraph{直観的説明}
フェルミ気体の低温比熱の直観的説明として、
フェルミ分布関数がステップ関数のような形から、\(k_BT\)だけ広がることから
上昇幅も\(k_BT\)である。
\begin{equation}
  \varDelta  U = D(\mu)(k_BT)(k_BT)
\end{equation}
つまり、個数も\(k_BT\)であり、エネルギー変化も\(k_BT\)であるので、絵からわかる。

ここから、熱は\(T^2\)に比例して、比熱は\(T\)に比例することがわかる。




% ここにgoodnoteの内容を入れる





磁場を入れたときの粒子のエネルギーは、
\begin{equation}
  \varepsilon_{\vec{k}\sigma} = \varepsilon_{\vec{k}} + \sigma \mu_B H
\end{equation}
となる。

\begin{figure}[h]
\centering


\tikzset{every picture/.style={line width=0.75pt}} %set default line width to 0.75pt        

\begin{tikzpicture}[x=0.75pt,y=0.75pt,yscale=-1,xscale=1]
%uncomment if require: \path (0,235); %set diagram left start at 0, and has height of 235

%Shape: Axis 2D [id:dp019695550918759075] 
\draw  (213,182.65) -- (436,182.65)(324,63) -- (324,229.17) (429,177.65) -- (436,182.65) -- (429,187.65) (319,70) -- (324,63) -- (329,70)  ;
%Straight Lines [id:da03717739112964291] 
\draw    (203,100) -- (441,100) ;
%Straight Lines [id:da6146232228002211] 
\draw    (252,60.17) -- (252,34.17) ;
\draw [shift={(252,32.17)}, rotate = 90] [color={rgb, 255:red, 0; green, 0; blue, 0 }  ][line width=0.75]    (10.93,-3.29) .. controls (6.95,-1.4) and (3.31,-0.3) .. (0,0) .. controls (3.31,0.3) and (6.95,1.4) .. (10.93,3.29)   ;
%Straight Lines [id:da7839153232997202] 
\draw    (388,30) -- (388,57.17) ;
\draw [shift={(388,59.17)}, rotate = 270] [color={rgb, 255:red, 0; green, 0; blue, 0 }  ][line width=0.75]    (10.93,-3.29) .. controls (6.95,-1.4) and (3.31,-0.3) .. (0,0) .. controls (3.31,0.3) and (6.95,1.4) .. (10.93,3.29)   ;
%Curve Lines [id:da8448149574349508] 
\draw    (193,45.17) .. controls (204,99.17) and (235,144.17) .. (324,146) ;
%Curve Lines [id:da29542722436391444] 
\draw    (325,211) .. controls (406,200.17) and (411,125.17) .. (416,58.17) ;

% Text Node
\draw (329,131.4) node [anchor=north west][inner sep=0.75pt]    {$\mu _{B} H$};
% Text Node
\draw (271,191.4) node [anchor=north west][inner sep=0.75pt]    {$-\mu _{B} H$};
% Text Node
\draw (314,38.4) node [anchor=north west][inner sep=0.75pt]    {$\varepsilon _{\vec{k} \sigma }$};
% Text Node
\draw (445,176.4) node [anchor=north west][inner sep=0.75pt]    {$D( \varepsilon )$};


\end{tikzpicture}

\caption{スピンの図}
\label{fig:スピンの図}
\end{figure}

\(\sigma\)スピンの電子数は、
\begin{align}
  N_{\sigma}& = \sum_{\vec{k}} f(\varepsilon_{\vec{k}\sigma}) \\
  & = \int_{0}^{\infty}\frac{1}{2}D(\varepsilon)f(\varepsilon + \sigma \mu_B H) \odif{\varepsilon}
\end{align}
となる。
ここから、状態密度を\(\mu\)の周りで展開してあげると\(D\)は偶数次しか出ててこないことを用いる。
\begin{align}
  N_{\sigma} & = \frac{N}{2} + \int_{0}^{\infty}\frac{1}{2}D(\varepsilon) \pdv{f(\varepsilon)}{\varepsilon}\Big|_{H=0} \sigma \mu_B H \odif{\varepsilon} + O(H^2)\\
  & = \frac{N}{2}  -  \frac{1}{2}\sigma \mu_B H \int_{0}^{\infty}\left[ D(\mu) + \frac{1}{2}D''(\mu)(\varepsilon-\mu)^2 \right]\left( \pdv{f(\varepsilon)}{\varepsilon} \right)_{H=0} \odif{\varepsilon} + O(H^2)\\
\end{align}
となる。ここで、ゾンマーフェルト展開と同様に、
\begin{equation}\label{eq:2.12}
  N_\sigma = \frac{N}{2} - \frac{1}{2}\sigma \mu_B H \left[ D(\mu) + \frac{1}{2}D''(\mu)\frac{\pi^2}{3}(k_BT)^2 \right] + O(H^2)
\end{equation}
となる。
よって磁化は、
\begin{align}\label{eq:2.13}
  M & = N_{\uparrow}\mu_{e \uparrow} - N_{\downarrow}\mu_{e \downarrow} \\
  & = -\mu_B \left( N_{\uparrow} - N_{\downarrow}\right) \\
  & = \mu_B^2 H \left[ D(\mu) + \frac{1}{2}D''(\mu) \frac{\pi^2}{3}(k_B T)^2 \right]
\end{align}
となるので、\(\vec{\mu}_e = - \mu_B \vec{\sigma}\)
ここで、式\eqref{eq:2.12}より、\(N_{\uparrow}+N_{\downarrow}=N+O(H^2)\)となる依って、
\begin{equation}
  \mu(H) = \mu(H=0) + O(H^2)
\end{equation}
帯磁率\(\chi = \pdv{M}{H}\Big|_{H=0}\) を求める際に\(\mu\)の\(H\)依存性を考慮する必要がない。
線形部分しか効かない。

\(T=0\)の場合の退治率は
\begin{equation}
  \chi = \pdv{M}{H} \Big|_{H=0} = \mu_B^2 D(\mu_0) >0
\end{equation}
となる。磁場をかけた方向に磁化することがわかる。
反対に磁化を書けて反対に磁化するものを反磁性と言ったりもするが、今回は常磁性体である。

次に有限温度の場合を考えてみよう。この場合には、\(\mu(T) = \mu_0 + \varDelta \mu\)を式\eqref{eq:2.13}に代入して、
\begin{equation}
  M = \mu_B^2 H \left[ D(\mu_0) +D'(\mu_0) \varDelta \mu + \frac{\pi^2}{6}(k_BT)^2 D''(\mu_0) \right] + O(T^4)
\end{equation}
となる。ここで、\eqref{eq:2.7}より、
\begin{equation}
  \varDelta \mu = - \frac{D'(\mu_0)}{D(\mu_0)}\frac{\pi^2}{6}(k_BT)^2 
\end{equation}
を代入して、

\begin{equation}
  M = \mu_B^2 H D(\mu_0) + \mu_B^2 H \frac{\pi^2}{6}(k_BT)^2 \left[ - \frac{D'(\mu)^2}{D(\mu)} + D''(\mu_0) \right]
\end{equation}
となる。より、\(\chi(T)=\chi(T=0) + \mu_B^2 \frac{\pi^2}{6}(k_BT)^2 \left[ - \frac{(D'(\mu_0))^2}{D(\mu_0)} + D''(\mu_0) \right] \)



3次元の場合には、\(D(\varepsilon)=c \varepsilon^{\frac{1}{2}}\)により、
\begin{equation}
  \chi(T) = \mu_B^2 D(\mu_0) \left[ 1 - \frac{\pi^2}{12}\left( \frac{k_BT }{\mu_0} \right)^2 \right]
\end{equation}
となる。
1次元の場合には、\(D(\varepsilon)=c \varepsilon^{-\frac{1}{2}}\)
となり、
\begin{equation}
  \chi = \mu_B^2 D(\mu_0) \left[ 1 + \frac{\pi^2}{12}\left( \frac{k_BT}{\mu_0} \right)^2 \right]
\end{equation}

\(T \gg T_F\)の場合、\(e^{\beta \mu}\ll 1\)より、\(\sigma\)スピンの分布関数
\begin{equation}
  f(\varepsilon+\sigma \mu_B H) = \exp\left( - \frac{\varepsilon-\mu + \sigma \mu_B H}{k_BT} \right)
\end{equation}
となる。
このようになるととても計算が簡単になり、\(\int_{0}^{\infty}D(\varepsilon)f(\varepsilon)\odif{\varepsilon}=N\)とすると、
\begin{align}
  N \sigma & = \int_{0}^{\infty} \frac{1}{2}D(\varepsilon) f(\varepsilon + \sigma \mu_B H) \odif{\varepsilon}\\
  & = \frac{1}{2}\int_{0}^{\infty}D(\varepsilon) f(\varepsilon) \left( 1 - \frac{\sigma \mu_B H}{k_BT} \right)\odif{\varepsilon} + O(H^2)\\
  & = \frac{N}{2}\left( 1 - \frac{\sigma \mu_B H}{k_BT} \right) + O(H^2)
\end{align}
となり、
\begin{equation}
  M = - \mu_B(N_{\uparrow} - N_{\downarrow}) =  \mu_B^2 H \frac{N}{k_BT} + O(H^2)
\end{equation}
となるために、
\begin{equation}
  \chi = \pdv{M}{H}\Big|_{H=0} =\frac{N\mu_B^2}{k_BT}
\end{equation}
となる。これを磁場が磁化にそろおうとする効果これをキュリー則という。

退治率の温度依存性をまとめると、
低温領域では、
\begin{equation}
  \begin{cases}
    \chi(T) = \mu_B^2 D(\mu_0) \left[ 1 - \frac{\pi^2}{12}\left( \frac{k_BT}{\mu_0} \right)^2 \right] & \text{3次元} \\
    \chi(T) = \mu_B^2 D(\mu_0) \left[ 1 + \frac{\pi^2}{12}\left( \frac{k_BT}{\mu_0} \right)^2 \right] & \text{1次元}
  \end{cases}
\end{equation}
となり、高温領域では、
\begin{equation}
  \chi(T) = \frac{N\mu_B^2}{k_BT}
\end{equation}
となる。

\paragraph{問題}これをグラフ\(\chi-T\)グラフを書け。


\paragraph{Wilson比}
Wilson比は、
\begin{equation}
  \frac{\chi(0)}{\gamma} = \frac{\mu_B^2 D(\mu_0)}{\frac{\pi^2}{3}k_B^2 D(\mu_0)} = \frac{3\mu_B^2}{\pi^2 k_B^2}
\end{equation}
となる。このような普遍的な定数になってしまう。

\paragraph{パウリ帯磁率の直観的導出}
パウリ帯磁率を直観的に導出しよう。

この時に斜線部分の面積は\(2\mu_BH\)
\begin{equation}
  M = - \mu_B (N_{\uparrow} - N_{\downarrow}) = \mu_B \frac{1}{2}D(\mu_0) \cdot 2 \mu_B H
\end{equation}
この時に、\(\frac{1}{2}D(\mu_0) \)はスピン当たりのDOSとなる。より、
\begin{equation}
  \chi = \mu_B^2 D(\mu_0)
\end{equation}
となる。これがパウリ帯磁率である。







\end{document}