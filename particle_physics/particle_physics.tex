\documentclass[titlepage]{ltjsarticle}
% ページスタイル
\pagestyle{headings}
% 数式
\usepackage{amsmath,amsfonts}
\usepackage{bm}
% 数式
\usepackage{amsthm}
\usepackage{amssymb}
\usepackage{amsfonts}
\usepackage{bm}
\usepackage{derivative}
% 式番号を途中でふらない(参照するものだけ参照する)
\usepackage{mathtools}
\mathtoolsset{showonlyrefs=true}
% 数式をすべてdisplaystyleにする。
\everymath{\displaystyle}
\usepackage{bookmark}
\hypersetup{unicode,bookmarksnumbered=true,hidelinks,final}
% mysimpleboxの設定
\usepackage{tcolorbox}
\tcbuselibrary{skins}
\newtcolorbox{mysimplebox}[1]{%
colframe=black,colback=white,
coltitle=black,colbacktitle=white,
boxrule=0.8pt,arc=0mm,
fonttitle=\sffamily\bfseries,
enhanced,
attach boxed title to top left={xshift=10mm ,yshift=-3mm},
boxed title style={frame hidden},
title=#1}
% Pythonでのグラフ描画をインポートするために必要なパッケージ
\usepackage{graphicx}
\usepackage[utf8]{inputenc}
\usepackage{pgfplots}
\pgfplotsset{compat=1.18}
\begin{document}
\title{素粒子物理学}
\author{名古屋大学理学研究科 飯島 轍}
\date{\today}
\maketitle


小テストとレポートの提出があった場合は、必ず出してください。
レポートを必ず出しましょう!

\section{素粒子物理学概論}

原子から原子核、核子、クォーク、レプトン、素粒子というように物質は階層的に構成されている。

電子ニュートリノには電荷のあるなしがある。
電荷が一高いなどの性質がある。
物質を構成しているスピンの要素は反整数のフェルミオンとなる。
素粒子の反応でバリオン数レプトン数の保存されることが知られていて、そのバリオン数を与える。このように何かしらの反応があった時にバリオン数が保存するようになっている。ゲージボゾンに関しては弱い作用グルーオンがありウィークボゾンについてはそれが電荷があることが知られている。このようにクォークとは何か?クォークとレプトンがあり、アップクォーク、ストレンジクォーク、アップクォークのように二つ存在するマイナス二分の一の電荷をもつ


第二三世代は大きく異なる点は一世代のコピーである。身の回りの性質はもっぱら1世代である。第四世代はないのか?のような疑問があるが、なぜ第四世台がないのかというのはなぜなのか?という疑問がある。このように、反粒子が存在している。

ニュートリノにも反ニュートリノが存在する。
このようにバリオン数などを考える必要がある。


ハドロン粒子がどんどん見つかった。そこでハドロンがわかると考えられる。反中性子が引っ付くと反重水素のようなものになる。

クォーク三つの状態をバリオンと呼ぶ。クォーク二つの場合にはメソンという。
これらがクォーク模型と呼ぶ。
分数電荷をもつ粒子の存在は受け入れられなかったが信じられるようになってきた。

自然界の相互作用は四つある。
これらは電磁気学、弱い相互作用、強い相互作用、重力相互作用である。
重力は極端に小さいために標準理論ではなかなか扱わない。これが弦理論などが出てくる。

物体があるとは何か?
まずは原子の塊である。これらは電気力により塊になっている。全体的には中性になっている。
重力が電磁気力と同じくらいの大きさの場合にはものを落とすうちにバラバラになるという事象が起こるだろう。

力はなぜ働くのか?
これらは交換によって生じる。ゲージボゾンが交換することにより力が生じる。

例えば、相互作用があった時には、フォトンの交換により説明することが出来る。
強い相互座用の時には、グルーオンであり、弱い相互作用の時には、ウィークボゾンの交換である。
電磁相互作用はフォトンの質量が0のために、無限に到達する。
弱い相互作用は非常に重いためにかなり近くないと相互作用を起こせないためである。強い相互作用はボゾンが自己相互作用を起こすために相互作用を動かすためである。

相互作用の到達距離\(d\)は
\begin{equation}
  \Delta d =  \frac{hc}{M}
\end{equation}
のように泡ラスことが出来る。
ウィークボゾンが存在している時には、エネルギーを保存していない。エネルギーの不確定性関係だけがある為である。到達距離は重たい粒子ほど短くなるt高尾がわかる。


強い相互作用をするクォークはカラー量子数を持つ。
三原色とのアナロジーからRGBと呼ばれる。
クォークはカラーをやり取りしてグルーオンと強い相互作用をする。
スピン上向きの三つする。これはフェルミ統計で禁止されつために、そのためにカラー量子数をどうにゅ数ることにより、カラー量子数によりあいけるする。RGBが合成されて白色になるように、そのように構成されていると考えられる。これは統計力学による過程である。

電磁相互作用はしません。
弱い相互作用は、荷電カレント反応という。相互作用によってフェルミオンの種類が変化する。

素粒子の質量はかなり持っている。

















\end{document}