\documentclass[titlepage]{ltjsarticle}
% ページスタイル
\pagestyle{headings}
% 数式
\usepackage{amsmath,amsfonts}
\usepackage{bm}
% 数式
\usepackage{amsthm}
\usepackage{amssymb}
\usepackage{amsfonts}
\usepackage{bm}
\usepackage{derivative}
% 式番号を途中でふらない(参照するものだけ参照する)
\usepackage{mathtools}
\mathtoolsset{showonlyrefs=true}
% 数式をすべてdisplaystyleにする。
\everymath{\displaystyle}
\usepackage{bookmark}
\hypersetup{unicode,bookmarksnumbered=true,hidelinks,final}
% mysimpleboxの設定
\usepackage{tcolorbox}
\tcbuselibrary{skins}
\newtcolorbox{mysimplebox}[1]{%
colframe=black,colback=white,
coltitle=black,colbacktitle=white,
boxrule=0.8pt,arc=0mm,
fonttitle=\sffamily\bfseries,
enhanced,
attach boxed title to top left={xshift=10mm ,yshift=-3mm},
boxed title style={frame hidden},
title=#1}
% Pythonでのグラフ描画をインポートするために必要なパッケージ
\usepackage{graphicx}
\usepackage[utf8]{inputenc}
\usepackage{pgfplots}
\pgfplotsset{compat=1.18}
% braket 記法を用いる。
\usepackage{physics2}
\usephysicsmodule{braket}
\newcommand{\mel}[3]{\braket[3]{#1}{#2}{#3}}
\newcommand{\ev}[1]{\braket[1]{#1}}
\begin{document}
\title{素粒子物理学}
\author{名古屋大学理学研究科 飯島 轍}
\date{\today}
\maketitle


小テストとレポートの提出があった場合は、必ず出してください。
レポートを必ず出しましょう!

\section{素粒子物理学概論}

原子から原子核、核子、クォーク、レプトン、素粒子というように物質は階層的に構成されている。

電子ニュートリノには電荷のあるなしがある。
電荷が一高いなどの性質がある。
物質を構成しているスピンの要素は反整数のフェルミオンとなる。
素粒子の反応でバリオン数レプトン数の保存されることが知られていて、そのバリオン数を与える。このように何かしらの反応があった時にバリオン数が保存するようになっている。ゲージボゾンに関しては弱い作用グルーオンがありウィークボゾンについてはそれが電荷があることが知られている。このようにクォークとは何か?クォークとレプトンがあり、アップクォーク、ストレンジクォーク、アップクォークのように二つ存在するマイナス二分の一の電荷をもつ


第二三世代は大きく異なる点は一世代のコピーである。身の回りの性質はもっぱら1世代である。第四世代はないのか?のような疑問があるが、なぜ第四世台がないのかというのはなぜなのか?という疑問がある。このように、反粒子が存在している。

ニュートリノにも反ニュートリノが存在する。
このようにバリオン数などを考える必要がある。


ハドロン粒子がどんどん見つかった。そこでハドロンがわかると考えられる。反中性子が引っ付くと反重水素のようなものになる。

クォーク三つの状態をバリオンと呼ぶ。クォーク二つの場合にはメソンという。
これらがクォーク模型と呼ぶ。
分数電荷をもつ粒子の存在は受け入れられなかったが信じられるようになってきた。

自然界の相互作用は四つある。
これらは電磁気学、弱い相互作用、強い相互作用、重力相互作用である。
重力は極端に小さいために標準理論ではなかなか扱わない。これが弦理論などが出てくる。

物体があるとは何か?
まずは原子の塊である。これらは電気力により塊になっている。全体的には中性になっている。
重力が電磁気力と同じくらいの大きさの場合にはものを落とすうちにバラバラになるという事象が起こるだろう。

力はなぜ働くのか?
これらは交換によって生じる。ゲージボゾンが交換することにより力が生じる。

例えば、相互作用があった時には、フォトンの交換により説明することが出来る。
強い相互座用の時には、グルーオンであり、弱い相互作用の時には、ウィークボゾンの交換である。
電磁相互作用はフォトンの質量が0のために、無限に到達する。
弱い相互作用は非常に重いためにかなり近くないと相互作用を起こせないためである。強い相互作用はボゾンが自己相互作用を起こすために相互作用を動かすためである。

相互作用の到達距離\(d\)は
\begin{equation}
  \Delta d =  \frac{hc}{M}
\end{equation}
のように泡ラスことが出来る。
ウィークボゾンが存在している時には、エネルギーを保存していない。エネルギーの不確定性関係だけがある為である。到達距離は重たい粒子ほど短くなるt高尾がわかる。


強い相互作用をするクォークはカラー量子数を持つ。
三原色とのアナロジーからRGBと呼ばれる。
クォークはカラーをやり取りしてグルーオンと強い相互作用をする。
スピン上向きの三つする。これはフェルミ統計で禁止されつために、そのためにカラー量子数をどうにゅ数ることにより、カラー量子数によりあいけるする。RGBが合成されて白色になるように、そのように構成されていると考えられる。これは統計力学による過程である。

電磁相互作用はしません。
弱い相互作用は、荷電カレント反応という。相互作用によってフェルミオンの種類が変化する。

素粒子の質量はかなり持っている。


\section{Dirac方程式}
負のエネルギー解が出てしまう。ので、確率密度が負となってしまう。そこで相対論的に従うような方程式を作ることが出来ないだろうか?つまり、時間微分に対して線形な方程式を作らなければならない。
方程式が共変であるためには空間微分\(\nabla\)についても線形のはずである。

相対論は時間的に流れていて、そこに空間の一変化がある。相対論は時空


Dirac 方程式

Klein-Gordon 方程式は、負のエネルギー解をもち、確率密度が負となってしまう!
負の確率を避けて、相対論( \(E^2=p^2+m^2\) )に従う方程式を作れないか?
\(\rightarrow \partial / \partial t\) について線形の方程式を考えよう!
さらに、方程式が共変的であるためには \(\nabla\) についても線形のはず
\(\rightarrow\) 一般的な形は、

\begin{align*}
H \psi=(\boldsymbol{\alpha} \cdot p+\beta m) \psi
\end{align*}


と書け、相対論的なエネルギーと運動量の関係;

\begin{align*}
H^2 \psi=\left(p^2+m^2\right) \psi
\end{align*}


を満たす。
\(\rightarrow\)

\begin{align*}
\begin{gathered}
H^2 \psi=\left(\alpha_i P_i+\beta m\right)\left(\alpha_j P_j+\beta m\right) \psi=\left(p^2+m^2\right) \psi \\
\therefore \alpha_i^2=1, \quad \alpha_i \alpha_j+\alpha_j \alpha_i=0, \quad \alpha_i \beta+\beta \alpha_i=0, \quad \beta^2=1
\end{gathered}
\end{align*}

\(\rightarrow\)
- \(\alpha_1, \alpha_2, \alpha_3, \beta\) はすべて反交換

\begin{align*}
\text { - } \alpha_1^2=\alpha_2^2=\alpha_3^2=\beta^2=1
\end{align*}

\(\therefore \alpha_i\) と \(\beta\) は単なる数ではなくて( \(\psi\) に作用する)行列であり、 \(\psi\) は多成分列ベクトル である。


Dirac-Pauli 表現

式(4)を満たす最低次元の行列は 4 行 4 列(一義的には決まらない)。
Dirac-Pauli 表現 ;

\begin{align*}
\boldsymbol{\alpha}=\left(\begin{array}{cc}
0 & \boldsymbol{\sigma} \\
\boldsymbol{\sigma} & 0
\end{array}\right), \quad \beta=\left(\begin{array}{cc}
1 & 0 \\
0 & -1
\end{array}\right)
\end{align*}

\(\sigma\) (Pauli 行列);

\begin{align*}
\sigma_1=\left(\begin{array}{ll}
0 & 1 \\
1 & 0
\end{array}\right), \quad \sigma_2=\left(\begin{array}{cc}
0 & -i \\
i & 0
\end{array}\right), \quad \sigma_3=\left(\begin{array}{cc}
1 & 0 \\
0 & -1
\end{array}\right)
\end{align*}


Dirac 方程式を満たす 4 成分列ベクトル \(\psi\) を Dirac スピノールと呼ぶ。
Dirac 方程式の解は、Klein-Gordon 方程式でみた 2 つの正負のエネルギー解だけでな く 4 つある!

このように4行4列の行列を用いて表現することが出来る。
Dirac方程式の解は正負のエネルギー解だけでなく二つある。

Dirac方程式を4限部分ベクトルをmのちいて相対論になじむ形に変形する。
\begin{equation}
  i \gamma^{\mu} \partial_{\mu} \psi - m \psi = 0
\end{equation}
ここで、\(\gamma^{\mu}\)はDiracの\(\gamma\)行列であり、
\begin{equation}
  \gamma^{\mu} = \left( \beta,\beta \bm{\alpha} \right)
\end{equation}
のように書くことが出来て、Dirac方程式は列ベクトル\(\psi\)の4成分の連立方程式として、
\begin{equation}
  \sum_{i=1}^{4}\left\{ \sum_{i} (\gamma^{\mu})_{jk} \partial_{\mu} - m\delta_{lk} \right\}\psi_k = 0
\end{equation}
となっている。

4成分の方程式になっている。\(\psi\)は4成分あることになっている。

4成分のカレントについて考えよう。
エルミート共益について考えると、
\begin{equation}
  (i \gamma^{\mu} \partial_\mu -m )\psi = 0
\end{equation}
となる。これをエルミート共役をとると。
\begin{equation}
  \partial_\mu\bar{\psi} (i \gamma^{\mu} -m ) = 0
\end{equation}

ここで、\(\bar{\psi}= \psi^{\dagger\gamma^0}\)を随伴スピノールという。

\(\rightarrow \bar{\psi} \times\) 式 (10)+式(11) \(\times \psi\) から、

\begin{align*}
\bar{\psi} \gamma^\mu \partial_\mu \psi+\left(\partial_\mu \bar{\psi}\right) \gamma^\mu \psi=\partial_\mu\left(\bar{\psi} \gamma^\mu \psi\right)=0
\end{align*}

\(\rightarrow\)

\begin{align*}
j^\mu=\bar{\psi} \gamma^\mu \psi
\end{align*}


は連続の式を満たし、確率密度;

\begin{align*}
\rho \equiv j^0=\bar{\psi} \gamma^0 \psi=\psi^{\dagger} \psi=\Sigma_{i=1}^4\left|\psi_i\right|^2
\end{align*}


は正値確定となる!
電子の 4 元電流密度は

\begin{align*}
j^\mu=-e \bar{\psi} \gamma^\mu \psi
\end{align*}


と書ける。


\subsection{自由粒子のスピノール}
自由粒子の場合には、
\begin{equation}
  \psi = u(p) e^{-ipx}
\end{equation}
の方とをもつDirac方程式の固有関数\(u(\bm{p})\)を求める。
この時に、
\begin{equation}
  Hu = (\bm{\alpha} \cdot \bm{p} + \beta m)u = Eu
\end{equation}
となる。ここで静止した粒子の場合には、
\begin{equation}
  Hu = \beta mu = 
  \begin{pmatrix}
    mI & 0 \\
    0 & -mI
  \end{pmatrix}u
\end{equation}
となることから、

エネルギー固有値は\(E= m,m,-m,-m\)であるので、
\begin{equation}
  u = 
  \begin{pmatrix}
    1 \\
    0 \\
    0 \\
    0
  \end{pmatrix}, 
  \begin{pmatrix}
    0 \\
    1 \\
    0 \\
    0
  \end{pmatrix},
  \begin{pmatrix}
    0 \\
    0 \\
    1 \\
    0
  \end{pmatrix},
  \begin{pmatrix}
    0 \\
    0 \\
    0 \\
    1
  \end{pmatrix}
\end{equation}

となる。

\(p \neq 0\) の場合は、簡単のために、粒子が z 方向に運動している場合を考えると、Dirac方程式は、

\begin{align*}
\begin{gathered}
H u=\left(\begin{array}{cc}
m & \sigma_3 P_3 \\
\sigma_3 P_3 & -m
\end{array}\right)\binom{u_A}{u_B}=E\binom{u_A}{u_B} \\
\sigma_3 P_3 u_B=(E-m) u_A, \\
\sigma_3 P_3 u_A=(E+m) u_B
\end{gathered}
\end{align*}


となり、 \(E>0\) の二つの解は、

\begin{align*}
u^1=N\left(\begin{array}{c}
1 \\
0 \\
\frac{P_3}{E+m} \\
0
\end{array}\right), \quad u^2=N\left(\begin{array}{c}
0 \\
1 \\
0 \\
\frac{-P_3}{E+m}
\end{array}\right)
\end{align*}

\(E<0\) の二つの解は、

\begin{align*}
u^3=N\left(\begin{array}{c}
\frac{-P_3}{|E|+m} \\
0 \\
1 \\
0
\end{array}\right), \quad u^4=N\left(\begin{array}{c}
0 \\
\frac{P_3}{|E|+m} \\
0 \\
1
\end{array}\right)
\end{align*}


と求まる。
4 つの解が直交していることは明らかである。


さて、っここから、
反粒子について出てきて、負の確率密度を回避することが出来たが負のエネルギー解を出すことが出来る。
二つの負のエネルギー解\(u^{(3,4)}\)は韓流に対応している。

ここで、\(E<0\)の場合には、
\begin{equation}
  \psi = e^{-i(-E)(-t)}
\end{equation}
と書くことが出来るので負のエネルギー解は時間を逆行する反粒子に考える。


Dirac方程式は正負のエネルギー解のそれぞれに打ちて二つの解を持っている。
これは粒子のスピン状態に対応する。

ヘリシティ演算子はスピンの運動量方向への射影を考えると、
\begin{equation}
  \frac{1}{2} \bm{\sigma} \cdot \bm{p} = \frac{1}{2}\sigma_3
\end{equation}
となり、
\begin{equation}
  \frac{1}{2}\bm{\sigma} \hat{\bm{p}} \chi^{(5)} = \frac{1}{2}\sigma_3 \chi^{(5)} = \pm \chi^{(5)}
\end{equation}
となる。ヘリシティ演算子の固有値hは\(\pm \frac{1}{2}\)の状態であり、スピンが右巻き・左巻きになっている。

反粒子の孫愛とヘリシティが自然な帰結として導かれた。




\end{document}