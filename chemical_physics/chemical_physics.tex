\documentclass{ltjsarticle}
% ページスタイル
\pagestyle{headings}
% 数式
\usepackage{amsmath,amsfonts}
\usepackage{bm}
% 数式
\usepackage{amsthm}
\usepackage{amssymb}
\usepackage{amsfonts}
\usepackage{bm}
\usepackage{derivative}
% 式番号を途中でふらない(参照するものだけ参照する)
\usepackage{mathtools}
\mathtoolsset{showonlyrefs=true}
% 数式をすべてdisplaystyleにする。
\everymath{\displaystyle}
\usepackage{bookmark}
\hypersetup{unicode,bookmarksnumbered=true,hidelinks,final}
% mysimpleboxの設定
\usepackage{tcolorbox}
\tcbuselibrary{skins}
\newtcolorbox{mysimplebox}[1]{%
colframe=black,colback=white,
coltitle=black,colbacktitle=white,
boxrule=0.8pt,arc=0mm,
fonttitle=\sffamily\bfseries,
enhanced,
attach boxed title to top left={xshift=10mm ,yshift=-3mm},
boxed title style={frame hidden},
title=#1}
% Pythonでのグラフ描画をインポートするために必要なパッケージ
\usepackage{graphicx}
\usepackage[utf8]{inputenc}
\usepackage{pgfplots}
\pgfplotsset{compat=1.18}
\begin{document}

\title{化学物理学}
\author{名古屋大学理学部 槇 瓦介}
\date{\today}

\maketitle

\tableofcontents

\section*{オリエンテーション}
% 教員紹介
\begin{mysimplebox}{教員紹介}
\begin{itemize}
\item 担当教員: 槇 瓦介
\item メールアドレス: maki.koske.v8@f.mail.nagoya-u.ac.jp
\item 連絡先:
\item オフィス
\item オフィスアワー:
\end{itemize} 
\end{mysimplebox}
メッセージを送る際にはメッセージの優先度を必ず高にしてください。
そのメッセージが遅れない可能性があります。

取り下げの場合は、制度適用

レポートを提出してください。未提出ではWになります。


\part{化学平衡(equibrility)}
巨視的にはもはや動かないもの
\section{熱力学の復習(一成分系)}
平衡状態が別の平衡状態に至るのだが、その過程はどのような操作をしてもいい。

統計力学の正当性は熱力学によるものである。熱力学は私たちの経験と合致する。
これが熱力学である。

なるべく一本道で考えられるようにする。
\subsection{平衡状態(equilibirium status eg status)における系の記述}
平衡状態をeq状態と略記することがある。

対象とする系としては、何らかに仕切られた系のことである。
\begin{description}
  \item[単純な系] 内部の仕切りなし(内部束縛壁無し)の系を考える。
  \item[複合系]  複数の単純系の組み合わせである。
\end{description}
系の記述は示量変数である体積\(V\)と物質量\(N\)で行う。
二個並べたら二倍になる量のことである。
\(X=(V,N)\)としよう。

環境とは熱力学的な系とする。

\begin{description}
  \item[前提1] ある環境と壁が存在する。この時に長時間経つと環境に依存して、系の状態が変化することがある。
長時間経つと環境に依存してeq状態に(時間がたっても系の巨視的な性質不変)同じ環境下のeq状態は\(X\)(示量変数の組)で決まる。
  \item[前提2] 環境を特徴づける示強変数:温度\(T\)の存在する。\((T:X)\rightarrow \)eq状態を決める。
  \item[前提3] ある種類のの壁が存在し(断熱壁)が存在すれば、長時間で平行状態になる。しかしその平行状態は環境の影響を受けない。(isolated) 系のT:系の初期状態で決まる
\end{description}

熱力学的な系の外にはマクロな力学系(外界)が存在する。
外からの操作から仕事を測定することが可能である。(例えば、壁を動かすこととか、壁の挿入や除去や流体の混合等)
温度制御を接触させることや断熱壁で囲むことが出来る。

系があり、環境があるだけではなかなか難しい。
典型的には、ピストンなどがついているとかを考える。

詳細の記述が必要か?というと、仕事を測定することが出来る。
とりもなおさず、測定可能になる。
壁を動かす必要がある。

壁を入れたり出したりすることが出来る。
これは壁に対して挿入や除去には仕事いらない。夏力学的な極限を考えると\(L^2\ll L^3\)より、仕事を無地できる。




\subsubsection{状態変数}
状態が決まるときに状態変数がきまる。
示量変数(エネルギー、\((U,V)\))と示強変数についての相加性がある。共通部分がないようにしてあげると、
\begin{equation}
  X(A+B) = X(A)+X(B)
\end{equation}
となる。
示量性があるときには、
\begin{equation}
  X(T;\lambda V, \lambda N) = \lambda X(T;V,N)
\end{equation}
となる。

示強変数とは、
\begin{equation}
  X(T;\lambda V , \lambda N) = X(T;V,N)
\end{equation}
となる。

過程としては、し状態と操作がありそのあとの状態から、始状態と終状態がある。

\subsubsection{等温過程}
等温過程では、\((T;V,N)\xrightarrow{i} (T;V,N)\)となり、等温環境下での過程である。

等温過程下では、力学的操作と熱浴との接触が存在する。
力学的過程であれば可能な操作であるならば何でもありである。

\subsubsection{断熱過程}
断熱過程では、\((T;V,N)\xrightarrow{ad} (T;V,N)\)となり、断熱環境下での過程である。

\subsubsection{広義の等温過程(extended isothermal process)}
\((T;V,N)\xrightarrow{ex} (T;V',N)\)となり、等温環境下での過程である。
温度\(T\)の環境下で、透熱とする。断熱壁をかぶせ断熱のまま操作をしよう。そのあとに、断熱壁をとり、\(T\)の等温環境下においておく。すると仮想的には等温での過程になる。

\subsubsection{準静的(guasistatic)過程}
状態が持続する過程のことをいう。平衡状態に近い状態間の一連の過程のことを「準静的過程」という。
これは、
\begin{equation}
  (T;V,N) \xrightarrow{qs} (T';V',N)
\end{equation}
となることがわかる。

逆の操作によって終状態から始状態に戻れる。
平衡状態から平衡状態への詳細について面倒を見るわけではない。
ものの量を一定にすることが出来る

で、系が外界にする仕事としては、
\begin{equation}
  W[(T;V,N)\xrightarrow{qs} (T';V',N)] = -\int_{(T;V,N)}^{(T';V',N)}dW
\end{equation}
と書くことができる。

系に流れる熱の量は、
\begin{equation}
  Q[(T;V)\xrightarrow{qs} (T';V')] 
\end{equation}

このようにどのような過程を通ったか?によって、仕事が変化する。つまり、過程によって仕事が変わる。

特に、
\begin{equation}
  (T,V)\xrightarrow{qs} (T+\varDelta T,V+\varDelta V)
\end{equation}
のような微小過程であり圧力が定義され続けている場合には、
\begin{equation}
  W[(T,V) \xrightarrow{qs} (T+\varDelta T,V+\varDelta V)] \simeq P\varDelta V
\end{equation}
このような場合には、準静的過程となり、さらに逆をたどることが出来る。
\begin{equation}
  W_S[(T',V')\xrightarrow{qs}(T,V)] = -W[(T,V)\xrightarrow{qs}(T',V')]
\end{equation}
のように負号が反対になったものになる。

\subsection{流体の熱力学的性質}
\begin{description}
  \item[前提4] 刑を構成する流体について、平衡状態における圧力\(P\) が\((T;V,N)\)で決まると仮定する。
  \item[前提5] 系を構成する流体に対して、定積熱容量が定まると仮定する。
\end{description}
\subsubsection{状態方程式}


\begin{equation}
  P = P(T;,V,N) 
\end{equation}
となり、理想気体の場合には、\(P= \frac{NRT}{V}\)となり、

\subsubsection{定積熱容量}
定積熱容量は、系と熱浴との接触によって、生じる\(T\)変化を考える。


系から熱\(Q\)が流れたとする。このときに、\(V,N\)が一定で、\(T\to T+ \varDelta T\)となるときに必要な\(Q\)を定積熱容量\(C_V\)となる。理想気体の時の例としては、\(c\)を定数として、
\begin{equation}
  C_V(T;V,N) = cNR
\end{equation}
熱の定義は温度が上がったら熱が流れたということにする。熱というのはエネルギーの流れであるというスタンスで記述している。


\subsection{熱力学の第一法則}
\subsubsection{仕事と熱の等価性}
仕事と熱であり、比例同じ量の異なる表現であり、エネルギーの流れとしての等価性がある。

\begin{description}
  \item[前提1.7] i過程版カルノーの原理
\end{description}
ケルビンの原理がある。

ケルビンの定理により、カルノー機関の効率性を見出すことが出来る。
証明をしよう。

\paragraph{証明}
二段階に分けて考える。
\begin{enumerate}
  \item 最高効率\(\eta_c\)の期間であるカルノー機関を見つける
  \item \(\eta\)より小さい機関の効率を示そう。
\end{enumerate}
過程1については\((T_+,V_0)\to (T_+;V_1)\)\(Q[T_+; V_0 \to V_1]\)となる。
カルノー機関としては、1qsなら\(W\)は最大になる。さらに\(T_+\)でサイクル\(W<0\)となり、\(T_+\)から\(T_-\)へ、

過程2としては、\((T_+;V_1)\to (T_-;V_3)\)となる。この過程は断熱過程である。あとは、過程3について、\((T_-,V_3)\to (T_-,V_2)\)と\((T_-,V_2)\to (T_+,V_0)\)となる。そこからカルノー機関\(C[T_+,T_-,V_0,V_1,V_2,V_3]\)
のみからな逆過程が実装できる。
熱機関の効率は、
\begin{equation}
  \eta = \frac{Q[T_+;V_0\to V_1]+ Q[T_-;V_3\to V_2]}{Q[T_+;V_0\to V_1]}
\end{equation}
であることから、\(\eta_c\)は、\(T_+,T_-\)のみによって決まり、物質の種類\(N,V_0,V_1,V_2,V_3\)に依らないことがわかる。
任意の\(T_+,T_-\)の熱浴を持つ任意の機関の効率\(\eta\)は\(\eta_c\)を超えない。ほかの任意の機関の効率は\(\eta_c\)を超えない。

方針は\(D+\alpha \times \text{逆カノニカル}\)
\(T_+\)との熱のやり取りなし、とするとケルビンの定理より、\(D\)は\(T_+\)との熱のやり取りをする。
\(Q_{D_+}\)は、\(T_+\)熱浴から\(D\)が吸収する熱量。\(Q_{Q_+}\)は\(D\)がする仕事であり、

逆カルノーとしては、
\begin{equation}
  \begin{cases}
    Q[T_+;V_0\xrightarrow{iqs}V_1] :  T_+\text{に放出する熱量} \\
    \eta_c : Q[T_+;V_0 \xrightarrow{iqs}V_1] : 逆カルノーのされる仕事
  \end{cases}
\end{equation}
このようにすると、逆カルノーのサイズの調整として、\(D_+\)として、示量性湯おr、
\begin{equation}
  Q_{D_+} = Q[T_+,\alpha V_0 \xrightarrow{iqs} \alpha V_1] = \alpha Q[T_+;V_0\xrightarrow{iqs}V_1]
\end{equation}
となるために
\begin{equation}
  \alpha = \frac{Q_{D+}}{Q[T_+,V_0 \xrightarrow{iqs}V_1]}
\end{equation}
となり、\(T_-\)が熱浴ならば、吸収する仕事の量は、
\begin{align}
  W_{total} & = \eta  Q_{D+} - \eta Q[T_+;V_0\xrightarrow{iqs}V_1]\\
  & = \eta Q_{D_+} - \eta Q_{D_+} \le 0
\end{align}
となることから、\(\eta \le \eta_c\)

効率\(\eta_c\)を超える2-T機関は損z内しないことを考える。
カルノーサイクルを考えるときに\(\eta'\le \eta_c\)と\(\eta_c\le \eta'\)となることから、カルノーサイクルは一意に定まることが言える。

カルノーの定理から、次の定理が言える。
\paragraph{定理1.4}2つのカルノー機関\(C[T_+,T_-;V_0,V_1,V_2,V_3;N]C[T_+,T_-,V_1',V_2',V_3',N']\)について、
\begin{equation}
  \frac{Q[T_-;V_2\xrightarrow{iqs}V_3]}{Q[T_+;V_0\xrightarrow{iqs}V_1]} = \frac{Q[T_+;V_0\xrightarrow{iqs}V_1']}{Q[T_-;V_2'\xrightarrow{iqs}V_3']}
\end{equation}

\paragraph{定義1.2}
絶対温度が\(T>T_\ast\)の時、
\begin{equation}
  \Theta(T) = \frac{Q[T;V_0\xrightarrow{iqs} V_1]}{Q[T_*:V_{0\ast}\xrightarrow{iqs}V_{1\ast}]}\Theta(T)_\ast  
\end{equation}
となる。
ここで、
\(\Theta_\ast\)と\(T_\ast\)を対応させる。
すると\(\Theta_\ast\)と\(T_\ast\)は同値になる。
以下もこれまで同様に絶対温度に対して\(T\)をとろう。
\paragraph{定理1.5}
絶対温度\(T\)と\(Q\)任意のカルノー機関\(C[T_+,T_-,V_0,V_1,V_2,V_3]\)に対して、
\begin{equation}
  \frac{Q[T_0;V_0\xrightarrow{iqs}V_1]}{T_+} = \frac{Q[T_+;V_0\xrightarrow{iqs}V_1]}{T_-}
\end{equation}
となる。そもそも\(Q,W\)は異なる。


\subsection{エントロピー}
エントロピーは、\(\varepsilon_g\)状態から\(\varepsilon_g\)状態へ\(\alpha\)過程を経てrたどり着ける状態を順序関係・不等式を用いて表す。

\subsubsection{不可逆過程}
言葉の曖昧さを取り払うために\(a\)過程に限ろう。
\paragraph{定義1.3}不可逆過程にある\(\alpha\)過程について、始状態と終状態とを逆にした\(\alpha\)過程が可能ならば、可逆過程である。

可逆過程ではない過程を
不可逆過程\((T,V)\xrightarrow{a}(T',V')\)かつ\(X((T',V')\xrightarrow{a}(T,V))\)となる。





\part{科学反応速度論(kinetics)}




\part{原子分子の構造とその周辺}
どんな対称性があるか?についての論


% 参考文献
\begin{thebibliography}{9}
\bibitem{1} マッカリー・サイモン「物理化学」
\bibitem{2} アトキンス「物理化学」
\bibitem{3} Charles R Catntor Paul R Schimmel Biological Chemistry W.F.Freeman Co 
\bibitem{4} 佐々 「熱力学」
\end{thebibliography}


\end{document}